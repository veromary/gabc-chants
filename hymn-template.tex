% !TEX TS-program = lualatex
% !TEX encoding = UTF-8
% usual packages loading:
%\usepackage{luatextra}
%\usepackage{graphicx} % support the \includegraphics command and options
\usepackage{geometry} % See geometry.pdf to learn the layout options. There are lots.
\geometry{letterpaper} % or letterpaper (US) or a5paper or....
\usepackage[autocompile]{gregoriotex} % for gregorio score inclusion

% Just the changes

%%%%%%%%%%%%%%%%%%%
%% And finally, all the spacings:
%%%%%%%%%%%%%%%%%%%

%This count tells gregoriotex what \gre@factor the below values correspond to.
%% All the following values correspond to a gre@factor of 17.  At this size the scores should be approximately the size usually seen in a gradual.
%% If you’re creating your own space configuration file, you may set this to some other value, should you so desire.
\greconffactor=16%

%How thick the lines should be.  When set equal to \greconffactor (above) the staff lines will be their default thickness.  Larger numbers result in thicker lines.
\grechangestafflinethickness{16}%


%%%%%%%%%%%%%%%%%%
% vertical spaces
%%%%%%%%%%%%%%%%%%

% the space for the translation
\grechangedim{translationheight}{0.5 cm}{scalable}%
%the space above the lines
\grechangedim{spaceabovelines}{0 cm}{scalable}%
% this counter is the threshold above which we start accounting notes above
% lines for additional space above lines. For instance with a threshold of
% 2 and a staff of 4 lines, notes with a pitch of k and l will not interfere
% with the space above lines
\grechangecount{additionaltopspacethreshold}{2}%
% same, for notes taken into account for alt text vertical position
\grechangecount{additionaltopspacealtthreshold}{0}%
% same, for notes taken into account for nabc vertical position
\grechangecount{additionaltopspacenabcthreshold}{4}%
%the space between the lines and the bottom of the text Changed
\grechangedim{spacelinestext}{0.5 cm}{scalable}%
%the per-note additional space between lines and the bottom of the text
\grechangedim{noteadditionalspacelinestext}{0.14413 cm}{scalable}%
% this counter is the number of low notes which will add on the
% noteadditionalspacelinestext.  For instance, with a threshold of 2, every
% note below c will add noteadditionalspacelinestext space for each pitch needed
% below c, accounting for the various signs.
\grechangecount{noteadditionalspacelinestextthreshold}{2}%
%the space beneath the text
\grechangedim{spacebeneathtext}{0 cm}{scalable}%
% height of the text above the note line
\grechangedim{abovelinestextraise}{-0.1 cm}{scalable}%
% height that is added at the top of the lines if there is text above the lines (it must be bigger than the text for it to be taken into consideration)
\grechangedim{abovelinestextheight}{0.3 cm}{scalable}%
% an additional shift you can give to the brace above the bars if you don't like it


%%%%%%%%%%
%% Line spacings
%%%%%%%%%%
\grechangedim{parskip}{1pt plus 1pt}{scalable}%
\grechangedim{lineskip}{0pt plus 1pt}{scalable}%
\grechangedim{baselineskip}{52pt plus 5pt minus 5pt}{scalable}%
\grechangedim{lineskiplimit}{0pt}{scalable}%

\usepackage[libertine]{../mypackage}
% to change the font to something better, you can install the kpfonts package (if not already installed). To do so
% go open the "TeX Live Manager" in the Menu Start->All Programs->TeX Live 2010
\grechangestaffsize{17}

%\def\greinitialformat#1{{\fontsize{37}{37}\selectfont #1}}
%small > footnotesize > scriptsize > tiny

%\marginsize{35pt}{35pt}{30pt}{0pt}
\usepackage{calc}
\geometry{outer=35pt,inner=35pt,top=30pt+\headsep+\headheight,bottom=\footskip}
\pagestyle{fancy} % no header or footers

% here we begin the document
\begin{document}
\chead{\heading}

% Here we set the initial font. Change 43 if you want a bigger initial.
\def\greinitialformat#1{%
{\fontsize{\hymninitialsize}{\hymninitialsize}\selectfont #1}%
}

% We set red lines here, comment it if you want black ones.
%\redlines

% We set VII above the initial.
{\def\annot{\small{Hymn.}}
\def\annottwo{\small{\hymnlinetwo}}
\setinitialspacing{\hymninitial}

% We type a text in the top right corner of the score:
%\commentary{{\small \emph{Cf. Is. 30, 19 . 30 ; Ps. 79}}}

% and finally we include the score. The file must be in the same directory as this one.
\prehymn
\gregorioscore{\hymntex}
\ifx\posthymn\undefined\else\posthymn\fi
}
\normalsize\vspace{0pt minus 20pt}
\selectlanguage{american}
\begin{multicols}{2}
\begin{psalmverses}
{\hymntranslation}
\end{psalmverses}
\end{multicols}
\selectlanguage{latin}
\smallskip
\ifx\vrtex\undefined%
\ifx\vr\undefined\else{\Vbar{}~\vr{}.

}\fi
\else{\vspace{0pt plus 18pt}%
{\greblockcustos\gregorioscore{\vrtex}}%
\vspace{0pt plus 20pt}
\ifx\prevrtranslation\undefined\else\prevrtranslation\fi%
\translation[]{%
\Vbar{}~\vtranslation{}%
\ifx\vrnospace\undefined%
\\%
\else\if\vrnospace T%
\hspace{7ex minus 3ex}%
\else%
\\%
\fi\fi%
\Rbar{}~\rtranslation{}%
}%
}%
\fi%

\ifx\benedicamusdominotex\undefined\else
\medskip
\emph{Magnificat}. \emph{Collect}.
\medskip

\def\annot{\small{\benedicamusdominotone}}
%\def\annottwo{}
\setinitialspacing{B}
\gresetfirstlineaboveinitial{\annot}{\annot}
\setsecondannotation{}
\setspaceafterinitial{0mm}
\setspacebeforeinitial{0mm}
\ifx\benedicamusdominointro\undefined\else
\hspace{3ex}\emph{\benedicamusdominointro:}\\
\fi
\gregorioscore{\benedicamusdominotex}
\ifx\benedicamusdominotexalt\undefined\else
{%
\vspace{0pt plus 12pt}
\ifx\benedicamusdominoaltintro\undefined
\def\benedicamusdominoaltintro{or}\fi
\hspace{3ex}\emph{\benedicamusdominoaltintro:}\\
\def\annot{\small{\benedicamusdominotonealt}}
%\def\annottwo{}
\setinitialspacing{B}
\gresetfirstlineaboveinitial{\annot}{\annot}
\setsecondannotation{}
\setspaceafterinitial{0mm}
\setspacebeforeinitial{0mm}
\greblockcustos%
\gregorioscore{\benedicamusdominotexalt}
}
\fi
\fi


\end{document}
